\documentclass[onecolumn]{article}

\usepackage{amsmath}
\usepackage{mathtools}
\usepackage{verbatim}
\usepackage{microtype}
\usepackage{booktabs}
\usepackage{hyperref}
\usepackage[margin=3cm]{geometry}
\usepackage{caption}
\usepackage{subcaption}
\usepackage{hyperref}

\newcommand{\E}[1]{\ensuremath{\times 10^{#1}}}
\newcommand{\D}[2]{\ensuremath{\frac{\partial #1}{\partial #2}}}
\newcommand{\de}{\ensuremath{\mathrm{d}}}
\newcommand{\dee}{\ensuremath{\ \mathrm{d}}}

\title{\texttt{StarStruc.jl}: Stellar structure in Julia}
\author{Asher Wasserman}
\date{}

\begin{document}

\maketitle

\section{Introduction}

This document presents a description of a static stellar structure model implemented in \texttt{Julia}, a high-performance dynamical programming language.  The accompanying code can be found at \url{https://github.com/adwasser/StarStruc.jl}.  For carefully considered initial guesses as to the boundary values, the code should converge on a structure model on the order of a few minutes.  For arbitrary initial guesses, the code will likely crash and burn.

\section{Physics}

Stars are fearsome beasts, so we tame them by introducing some simplifying assumptions.  We model stars as non-rotating, time-independent, uniform composition, spherically symmetric cows.  Such domesticated animals still can pose numerical dangers to reckless star-cow wranglers, however.

% Stellar Structure and Evolution, Kippenhahn, Weigert, and Weiss, 2012.

We can concisely write the equations of stellar structure as
\begin{align}
  \D{r}{m} &= \frac{1}{4\pi r^2 \rho} \label{eq:drdm} \\
  \D{\ell}{m} &= \epsilon_n  \label{eq:dldm} \\
  \D{P}{m} &= -\frac{Gm}{4\pi r^4} \label{eq:dPdm} \\
  \D{T}{m} &= -\frac{GmT}{4\pi r^4P} \nabla \label{eq:dTdm}
\end{align}
What is then left, is to specify the equation of state, $\rho(P, T)$, the energy generation per unit mass $\epsilon_n$, the temperature gradient, $\nabla = \partial\log T / \partial\log P$, and the opacity, $\kappa(\rho, T)$.

We assume that the star is a fully ionized ideal gas, so its equation of state is
\begin{equation}
  P(\rho, T) = \frac{\rho}{\mu m_H} kT + \frac{a_\text{rad}}{3} T^4
\end{equation}
where 
\begin{equation}
  \mu = \frac{1}{1 + 3X + Y/2}
\end{equation}
is the dimensionless mean molecular weight for hydrogen mass fraction, $X$, and helium mass fraction, $Y$.  The term on the right accounts for pressure due to radiation and is generally small compared to the gas pressure.

For simplicity, we model energy transport such that convection efficiently transports any energy beyond the adiabatic limit, and so
\begin{align} \nabla = \frac{T}{P}\D{T}{P} = 
  \begin{cases}
    \nabla_\text{ad} & \nabla_\text{rad} \ge \nabla_\text{ad} \\
    \nabla_\text{rad} & \nabla_\text{rad} < \nabla_\text{ad}
  \end{cases}
\end{align}
For our assumed ideal gas, $\nabla_\text{ad} = 0.4$, and for a diffusive radiative transport model,
\begin{equation} 
  \nabla_\text{rad} = \frac{3}{16\pi a c G}\frac{\kappa \ell P}{m T^4} 
\end{equation}

We only model hydrogen burning for energy generation.  Hydrogen burning occurs through the proton-proton chain and the CNO cycle, both of which have their own temperature dependencies.  In the weak electron shielding approximation, we have
\begin{align}  
  \epsilon_\text{pp} &= 2.5\E{4} \psi g_\text{11} \rho X_1^2 T_9^{-2/3} \exp\left(-\frac{3.381}{T_9^{1/3}}\right) \text{ erg s}^{-1}\text{ g}^{-1} \\
  \epsilon_\text{CNO} &= 8.24\E{25} g_{14,1} X_\text{CNO} X_1 \rho T_9^{-2/3} \exp\left(-15.231 T_9^{-1/3} - \left(\frac{T_9}{0.8}\right)^2\right) \text{ erg s}^{-1}\text{ g}^{-1} \nonumber
\end{align}
where $T_9 = 10^{-9} \times (T / K)$. Here, $\psi$ accounts for the different branches of the proton-proton chain, and is generally between 1 and 1.5. The polynomial terms $g_{11}$ and $g_{14, 1}$ are given by
\begin{align}
  g_{11} &= 1 + 3.82 T_9 + 1.51 T_9^2 + 0.144 T_9^3 - 0.0114 T_9^4 \\
  g_{14, 1} &= 1 - 2.00 T_9 + 3.41 T_9^2 - 2.43 T_9^3 \nonumber
\end{align}
For the CNO mass fraction, $X_\text{CNO}$, we assume that carbon, nitrogen, and oxygen make up 70\% of the metal mass fraction.

The last missing piece for completely specifying the derivatives is a relevant opacity model for stellar matter.  Here we use the OPAL opacity tables, taken from \url{http://opalopacity.llnl.gov/}, and interpolate between values of $\rho$ and $T$.

With all of the above, we have four coupled first-order differential equations.  To solve for our four structure profiles, we need to impose some physically meaningful boundary conditions.  Starting from a mass shell very close to the center, $m_c$, we can use the linearizations of equations \ref{eq:drdm} and \ref{eq:dldm} to get
\begin{align}
  r_c &= \left(\frac{3}{4\pi \rho_c}\right)^{1/3} m_c^{1/3} \\
  \ell_c &= \epsilon_n m_c \nonumber
\end{align}

At the surface, we define our photosphere to be at the radius, $R_s$, where we have an optical depth of $2 / 3$.  Then from hydrostatic equilibrium and the definition of the effective temperature, we have
\begin{align}
  P_s &= \frac{GM}{R_s^2}\frac{2}{3}\frac{1}{\kappa} \\
  L_s &= 4\pi R_s^2 \sigma T_s^4 \nonumber
\end{align}

\section{Numerics}

\section{Results}

\section{Conclusions}

\end{document}

%%% Local Variables:
%%% mode: latex
%%% TeX-master: t
%%% End:
