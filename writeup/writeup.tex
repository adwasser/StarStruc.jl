\documentclass[onecolumn]{article}

\usepackage{amsmath}
\usepackage{mathtools}
\usepackage{verbatim}
\usepackage{microtype}
\usepackage{booktabs}
\usepackage{hyperref}
\usepackage[margin=3cm]{geometry}
\usepackage{caption}
\usepackage{subcaption}
\usepackage{hyperref}

\newcommand{\E}[1]{\ensuremath{\times 10^{#1}}}
\newcommand{\D}[2]{\ensuremath{\frac{\partial #1}{\partial #2}}}
\newcommand{\de}{\ensuremath{\mathrm{d}}}
\newcommand{\dee}{\ensuremath{\ \mathrm{d}}}

\title{\texttt{StarStruc.jl}: Stellar structure in Julia}
\author{Asher Wasserman}
\date{}

\begin{document}

\maketitle

\section{Introduction}

This document presents a description of a static stellar structure model implemented in \texttt{Julia}, a high-performance dynamical programming language.  The accompanying code can be found at \url{https://github.com/adwasser/StarStruc.jl}.  For carefully considered initial guesses as to the boundary values, the code should converge on a structure model on the order of a few minutes.  For arbitrary initial guesses, the code will likely crash and burn.

\section{Background}

Stars are fearsome beasts, so we tame them by introducing some simplifying assumptions.  We model stars as non-rotating, time-independent, uniform composition, spherically symmetric cows.  Such domesticated animals still can pose numerical dangers to reckless star-cow wranglers, however.

We assume that the star is a fully ionized ideal gas, so its equation of state is
\begin{equation}
  P(\rho, T) = \frac{\rho}{\mu m_H} kT + \frac{a_\text{rad}}{3} T^4
\end{equation}
where 
\begin{equation}
  \mu = \frac{1}{1 + 3X + Y/2}
\end{equation}
is the dimensionless mean molecular weight for hydrogen mass fraction, $X$, and helium mass fraction, $Y$.  The term on the right accounts for pressure due to radiation and is generally small compared to the gas pressure.

With this equation of state, our simple equations of stellar structure are
\begin{align}
  \D{r}{m} &= \frac{1}{4\pi r^2 \rho} \\
  \D{\ell}{m} &= \epsilon_n \nonumber \\
  \D{P}{m} &= -\frac{Gm}{4\pi r^4} \nonumber \\
  \D{T}{m} &= -\frac{GmT}{4\pi r^4P} \nabla \nonumber
\end{align}

For simplicity, we model energy transport such that convection efficiently transports any energy beyond the adiabatic limit, and so
\begin{align} \nabla = \frac{T}{P}\D{T}{P} = 
  \begin{cases}
    \nabla_\text{ad} & \nabla_\text{rad} \ge \nabla_\text{ad} \\
    \nabla_\text{rad} & \nabla_\text{rad} < \nabla_\text{ad}
  \end{cases}
\end{align}
For our assumed ideal gas, $\nabla_\text{ad} = 0.4$, and for a diffusive radiative transport model,
\begin{equation} 
  \nabla_\text{rad} = \frac{3}{16\pi a c G}\frac{\kappa \ell P}{m T^4} 
\end{equation}

We model the energy generation only through hydrogen burning.  $\epsilon_n$  is the nuclear burning luminosity per unit mass, a quantity that is dependent on temperature, density and composition.  The hydrogen burning occurs through the proton-proton chain and the CNO cycle, both of which have their own temperature dependencies.

\begin{align}  
  \epsilon_\text{pp} =& 2.5\E{4} \psi f_{11} g_\text{11} \rho X_1^2 T_9^{-2/3} \\
                      & \times \exp\left(-\frac{3.381}{T_9^{1/3}}\right) \nonumber \\
  \epsilon_\text{CNO} =& 8.24\E{25} g_{14,1} X_\text{CNO} X_1 \rho T_9^{-2/3} \nonumber \\
                      & \times \exp\left(-15.231 T_9^{-1/3} - \left(\frac{T_9}{0.8}\right)^2\right) \nonumber
\end{align}


\section{Methods}

\section{Results}

\section{Conclusions}

\end{document}

%%% Local Variables:
%%% mode: latex
%%% TeX-master: t
%%% End:
